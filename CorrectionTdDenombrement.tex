\documentclass[12pt]{article}
\usepackage{stmaryrd}
\usepackage{graphicx}
\usepackage[utf8]{inputenc}

\usepackage[french]{babel}
\usepackage[T1]{fontenc}
\usepackage{hyperref}
\usepackage{verbatim}

\usepackage{color, soul}

\usepackage{pgfplots}
\pgfplotsset{compat=1.15}
\usepackage{mathrsfs}

\usepackage{amsmath}
\usepackage{amsfonts}
\usepackage{amssymb}
\usepackage{tkz-tab}
\author{Kadidiatou Diallo}
\title{\textbf{Correction}}
\date{\today}
\usepackage{tikz}
\usetikzlibrary{arrows, shapes.geometric, fit}

% Commande pour la couleur d'accentuation
\newcommand{\myul}[2][black]{\setulcolor{#1}\ul{#2}\setulcolor{black}}
\newcommand\tab[1][1cm]{\hspace*{#1}}

\begin{document}
\maketitle
\newpage

\section*{\underline{\textbf{\textcolor{red}{Exercice 6}}}} 

Parmi $40$ secrétaires : 

\begin{itemize}
    \item $8$ parlent russe,
    \item $15$ parlent anglais,
    \item $8$ parlent espagnol,
    \item $4$ parlent anglais et espagnol,
    \item $5$ parlent anglais et russe,
    \item $2$ parlent espagnol et russe,
    \item $2$ parlent les trois langues.
\end{itemize}

\begin{enumerate}
    \item[a.] Combien de secrétaires parlent au moins une des trois langues ?
    
    \item[b.] Combien de secrétaires ne connaissent aucune des trois langues ?
    
    \item Faire un diagramme.
\end{enumerate}

\subsection*{Solution}
On commence par traduire les données de l'énoncé en définissant :

l'ensemble  de toutes les secrétaires, donc  card(E)=40.

l'ensemble  de celles qui parlent russe, donc  card(R)=8.

l'ensemble  de celles qui parlent anglais, donc  card(A)=15.

l'ensemble  de celles qui parlent allemand, donc card(E)=8. 

De plus: card($A\cap E$)=4, card($A\cap R$)=5, card($E\cap R$)=2 et\\ card($A\cap E\cap R$)=2.

L'ensemble des secrétaires qui ne connaissent aucune de ces trois langues est $\overline{A}\cap \overline{E}\cap \overline{R}$ = $\overline{A\cup E\cup R}$.

Donc : card($\overline{A}\cap \overline{E}\cap \overline{R}$)=card(E)-card($A\cup E\cup R$)=40-card($A\cup E\cup R$)

D'après la formule du crible,\\
card($A\cup D\cup R$)=card(A)+card(E)+card(R)-card($A\cap E$)-card($A\cap R$)-card($E\cap R$)+card($A\cap E\cap R$).

Donc: card($A\cup E\cup R$)=15+8+8-4-5-2+2=22

Donc: card($\overline{A}\cap \overline{E}\cap \overline{R}$)=40-22=18

Conclusion : Il y a 18 secrétaires qui ne connaissent aucune des trois langues.

\subsection*{Diagramme de Venn}

\begin{center}
\begin{tikzpicture}
    \def\firstcircle{(0,0) circle (1.5)}
    \def\secondcircle{(1.5,0) circle (1.5)}
    \def\thirdcircle{(0.75,-1.5) circle (1.5)}
    \begin{scope}
        \clip \firstcircle;
        \fill[red!50] \secondcircle;
    \end{scope}
    \begin{scope}
        \clip \firstcircle;
        \fill[green!50] \thirdcircle;
    \end{scope}
    \begin{scope}
        \clip \secondcircle;
        \fill[blue!50] \thirdcircle;
    \end{scope}
    \draw \firstcircle node[below left] {$R$};
    \draw \secondcircle node[below right] {$A$};
    \draw \thirdcircle node[above] {$E$};
    \node at (0.75,0) {2};
    \node at (-0.5,0) {1};
    \node at (2.0,0) {3};
    \node at (0.75,-2.5) {0};
    \node at (1.25,-1) {4};
    \node at (-0.25,-1) {5};
    \node at (0.75,-0.5) {2};
    \node at (3,1) {18};
\end{tikzpicture}
\end{center}
\end{document}
