\documentclass[12pt]{article}
\usepackage{stmaryrd}
\usepackage{graphicx}
\usepackage[utf8]{inputenc}

\usepackage[french]{babel}
\usepackage[T1]{fontenc}
\usepackage{hyperref}
\usepackage{verbatim}

\usepackage{color, soul}

\usepackage{pgfplots}
\pgfplotsset{compat=1.15}
\usepackage{mathrsfs}

\usepackage{amsmath}
\usepackage{amsfonts}
\usepackage{amssymb}
\usepackage{tkz-tab}
\author{Destinés à la 1\textsuperscript{ère}S2\\Au Lycée de Dindéferlo}
\title{\textbf{Dénombrement et Probabilités}}
\date{\today}
\usepackage{tikz}
\usetikzlibrary{arrows, shapes.geometric, fit}

% Commande pour la couleur d'accentuation
\newcommand{\myul}[2][black]{\setulcolor{#1}\ul{#2}\setulcolor{black}}
\newcommand\tab[1][1cm]{\hspace*{#1}}

\begin{document}
\maketitle
\newpage
\section*{\underline{\textbf{\textcolor{red}{I. Dénombrement}}}}
\subsection*{\underline{\textbf{\textcolor{red}{1. Notion d’ensemble fini}}}}
Si on collectionne des objets quelconques (de même nature ou non) que l’on peut distinguer deux à deux 
par un critère alors on définit un ensemble. Chaque objet d’un ensemble est appelé élément de l’ensemble. 
Oralement : Un ensemble est souvent noté par une lettre majuscule tandis qu’un élément de cet 
ensemble est noté par une lettre minuscule. Le nombre d’éléments d’un ensemble \(E\) est dit cardinal 
de \(E\) et est noté \(\text{card} (E)\).
\subsection*{\underline{\textbf{\textcolor{red}{a. Exemple}}}}
Notons \(E\), l’ensemble des lettres du mot ``diallo'':
\begin{itemize}
    \item Les éléments de \(E\) sont les lettres : \(d\), \(i\), \(a\), et \(l\). On note \(E = \{d, i, a, l, o\}\) ou 
    \(E = \{d, l, a, o, i\}\) ou \(E = \{a, l, d, i, o\} = \cdots\)
    \item \(d\) est un élément de \(E\) donc on écrit \(d \in E\) % Coloration d'une expression spécifique
    \item Le cardinal de \(E\) est \(\text{card}(E) = 5\).
\end{itemize}
\subsection*{\underline{\textbf{\textcolor{red}{b. Remarque}}}}
Un ensemble qui ne contient aucun élément est dit ensemble vide et est noté \(\emptyset\). On a donc 
\(\text{card}(\emptyset) = 0\). Par exemple, l’ensemble \(E\) des élèves de la TL de 2024 du Lycée de 
Dindéfelo ayant pour nom de famille ``kanté'' est l’ensemble vide.
\subsection*{\underline{\textbf{\textcolor{red}{2. Parties d’un ensemble}}}}
\subsection*{\underline{\textbf{\textcolor{red}{a. Définition}}}}
Un ensemble \(A\) est une partie (ou sous-ensemble) d’un ensemble \(E\) si tous les éléments de \(A\) sont aussi des éléments de \(E\). Dans ce cas, on écrit \(A \subset E\).
\subsection*{\underline{\textbf{\textcolor{red}{b. Intersection de deux ensembles}}}}
Soient \( A \) et \( B \) deux parties d’un ensemble \( E \). L’intersection de \( A \) et \( B \) est la partie de \( E \) notée \( A \cap B \) dont les éléments sont ceux qui sont communs à \( A \) et \( B \).
Si \( A \) et \( B \) n’ont aucun élément en commun alors \( A \cap B = \emptyset \) et on dit que \( A \) et \( B \) sont disjoints.
\subsection*{\underline{\textbf{\textcolor{red}{c. Réunion de deux ensembles}}}}
La réunion de deux parties \( A \) et \( B \) d’un ensemble \( E \) est la partie de \( E \) notée \( A \cup B \) dont les éléments sont ceux qui sont soit dans \( A \) seulement, soit dans \( B \) seulement, soit dans \( A \) et \( B \). On a les propriétés suivantes :
\begin{itemize}
    \item \( \text{card}(A \cup B) = \text{card}(A) + \text{card}(B) - \text{card}(A \cap B) \)
    \item Si \( A \) et \( B \) sont disjoints alors la propriété devient\\ \( \text{card}(A \cup B) = \text{card}(A) + \text{card}(B) \)
\end{itemize}
\subsection*{\underline{\textbf{\textcolor{red}{d. Complémentaire d’une partie d’un ensemble}}}}
Soit \( A \) une partie de \( E \). Le complémentaire de \( A \) dans \( E \) est la partie de \( E \) notée \( \overline{A} \) dont les éléments sont ceux qui ne sont pas dans \( A \).\\

Oralement : Par exemple si \( E \) est l’ensemble des élèves de la TL, \( A \), l’ensemble des garçons de la TL alors \( \overline{A} \), le complémentaire de \( A \) dans \( E \) est l’ensemble des filles de la TL. On a les propriétés suivantes :
\begin{itemize}
    \item \( A \cap \overline{A} = \emptyset \)
    \item \( A \cup \overline{A} = E \)
    \item \( \text{card}(A) + \text{card}(\overline{A}) = \text{card}(E) \)
\end{itemize}
\subsection*{\underline{\textbf{\textcolor{red}{3. Outils de dénombrement}}}}
\subsection*{\underline{\textbf{\textcolor{red}{a. p-liste}}}}
\subsection*{\underline{\textbf{\textcolor{red}{i. Définition et exemple }}}}
Soit $E$ un ensemble de cardinal $n$ et $p$ un entier naturel tel que $p \geq 1$. Une p-liste d’éléments de $E$ est une liste (suite ordonnée) de $p$ éléments distincts ou non de $E.$
une liste est notée par des parenthèses.

Ainsi, la liste $(1,2)\neq (2,1)$

Dans une liste, un élément peut être représenté plusieurs fois. par exemple $(1,2,2)$

Considérons $E = \left\lbrace d; i; a ; l ; o\right\rbrace $ alors

\begin{itemize}
\item[•] \textcolor{teal}{$(d,i); (a,l) ; (l,o) ;\cdots$sont des 2-listes d’éléments de $E$.}
\item[•] \textcolor{teal}{$ffa ; fal ;aaa\cdots$ sont des 3-listes d’éléments de $E$.}
\item[•] \textcolor{teal}{$faal ; llff\cdots$ sont des 4-listes d’éléments de E.}
\end{itemize}
\subsection*{\underline{\textbf{\textcolor{red}{ii. Propriété}}}}
Soit \(E\) un ensemble de cardinal \(n\) et \(p\) un entier \(\geq 1\). Le nombre de p-listes d’éléments de \(E\) est \(n^{p}\).

Par exemple, si \(E = \{d, i,  a, l , o\}\) alors :
\begin{itemize}
    \item[•] \textcolor{teal}{Le nombre de 2-listes d’éléments de \(E\) est \(5^{2} = 25\)}
    \item[•] \textcolor{teal}{Le nombre de 3-listes d’éléments de \(E\) est \(5^{3} = 125\)}
    \item[•] \textcolor{teal}{Le nombre de 4-listes d’éléments de \(E\) est \(5^{4} = 625\)}
\end{itemize}

\textcolor{red}{Oralement :} Pour calculer \(n^{p}\) avec la calculatrice, on utilise la touche \(y^{x}\) de la façon suivante : on tape \(n\) sur la calculatrice d'abord, puis on appuie sur la touche \(y^{x}\), puis on tape \(p\) sur la calculatrice et enfin, on appuie sur la touche \(=\) et le résultat s’affiche.
\subsection*{\underline{\textbf{\textcolor{red}{iii. Tirages successifs avec remise}}}}
Considérons une urne qui contient \(n\) éléments et soit \(p\) un entier tel que \(p \geq 1\). Lorsqu'on tire successivement avec remise \(p\) éléments parmi ces \(n\) éléments de l’urne, le nombre de ces tirages successifs avec remise est égal au nombre de p-listes d’éléments pris parmi \(n\) éléments, c’est-à-dire \(n^{p}\).

\textcolor{red}{Oralement :} Ceci consiste à les tirer un à un mais en notant l’élément tiré puis en le remettant à chaque fois dans l’urne avant de tirer le suivant. À la fin des \(p\) tirages successifs, on obtient quelque chose que l’on peut identifier à une liste de \(p\) éléments distincts ou non issus de l’urne, c’est-à-dire une p-liste d’éléments de l’urne. Ainsi, le nombre de tirages successifs avec remise de \(p\) éléments pris parmi les \(n\) éléments d’une urne est \(n^{p}\).
\subsection*{\underline{\textbf{\textcolor{red}{b. Arrangement}}}}
\subsection*{\underline{\textbf{\textcolor{red}{i. Définition et exemple}}}}
Soit \( E \) un ensemble de cardinal \( n \) et \( p \) un entier naturel tel que \( 1 \leq p \leq n \). Un \textbf{arrangement} de \( p \) éléments de \( E \) est une liste (suite ordonnée) de \( p \) éléments distincts de \( E \).

\textbf{Par exemple}, si \( E = \{f, a, l\} \), alors:
\begin{itemize}
    \item \( fa, af, lf, \ldots \) sont des arrangements de 2 éléments de \( E \).
    \item \( fal, lfa, \ldots \) sont des arrangements de 3 éléments de \( E \).
\end{itemize}

\textbf{Oralement :} Comme \( E \) contient 3 éléments, il n’existe pas d’arrangements de plus de 3 éléments de \( E \).
\subsection*{\underline{\textbf{\textcolor{red}{ii. Propriété}}}}
Le nombre d'arrangements de \( p \) éléments pris parmi \( n \) éléments d'un ensemble est noté \( A_n^p \) (on lit "A n p") et est égal au produit des \( p \) entiers consécutifs dont le plus grand est \( n \). C'est-à-dire:
\[
A_n^p = n \times (n-1) \times \cdots \times (n-p+1)
\]

\textbf{Exemple:}
Soit \( E = \{f, a, l\} \) alors:
\begin{itemize}
    \item[•] Le nombre d'arrangements de 2 éléments de \( E \) est noté \( A_3^2 \) et est égal à \( A_3^2 = 3 \times 2 = 6 \).
    \item[•] Le nombre d'arrangements de 3 éléments de \( E \) est \( A_3^3 = 3 \times 2 \times 1 = 6 \).
\end{itemize}

\textbf{Calcul sur la calculatrice:}
Pour calculer \( A_n^p \) avec la calculatrice, on utilise la touche « nPr ».
\subsection*{\underline{\textbf{\textcolor{red}{iii.Tirages successifs sans remise}}}}
Le nombre de tirages successifs sans remise de p éléments dans une urne contenant n 
éléments est $A_{n}^{p}$
\subsection*{\underline{\textbf{\textcolor{red}{c. Combinaison}}}}
\subsection*{\underline{\textbf{\textcolor{red}{i. Définition et exemple }}}}
Soit $E$ un ensemble de cardinal $n$ et $p$ un entier naturel tel que $0 \leq p \leq n$. Une combinaison de 
$p$ éléments de $E$ est une partie de $E$ contenant $p$ éléments.

Par exemple si $E = \{f, a, l\}$ alors

$\{f\}$, $\{a\}$ et $\{l\}$ sont les combinaisons d'un élément de $E$.

$\{f, a\}$, $\{f, l\}$ et $\{a, l\}$ sont les combinaisons de 2 éléments de $E$.

$E = \{f, a, l\}$ est la seule combinaison de 3 éléments de $E$.
\subsection*{\underline{\textbf{\textcolor{red}{ii. Propriété }}}}
Le nombre de combinaisons de $p$ éléments pris parmi $n$ éléments d'un ensemble, noté $C_{n}^{p}$ (on lit $C_{n}^{p}$) est $C_{n}^{p} = \frac{A_{n}^{p}}{1 \times 2 \times 3 \times \cdots \times p}$.

Par exemple, si $E = \{f, a, l\}$ alors :
\begin{itemize}
  \item[•] Le nombre de combinaisons de 2 éléments de $E$ est $C_{3}^{2} = \frac{A_{3}^{2}}{1 \times 2} = 3$.
  \item[•] Le nombre de combinaisons de 3 éléments de $E$ est $C_{3}^{3} = \frac{A_{3}^{3}}{1 \times 2 \times 3} = 1$.
\end{itemize}
Pour calculer $C_{n}^{p}$ avec la calculatrice, on utilise la touche « n cr ».
\subsection*{\underline{\textbf{\textcolor{red}{iii. Tirages simultanés}}}}
\textbf{Tirages simultanés:} Le nombre de tirages simultanés de $p$ éléments dans une urne contenant $n$ éléments est $C_{n}^{p}$.

\textbf{Résumé:}

En dénombrement, si dans les objets à dénombrer, il:
\begin{itemize}
    \item[•] Y a de l’ordre et une possibilité de répétition d’éléments dans les objets alors on utilise les p-listes. Ainsi un tirage successif avec remise de $p$ éléments correspond à une p-liste.
    \item[•] Y a de l’ordre mais il n’y a pas de répétition d’éléments dans les objets alors on utilise les arrangements. Ainsi un tirage successif sans remise de $p$ éléments correspond à un arrangement.
    \item[•] n’y a pas d’ordre et il n’y a pas de possibilité de répétition d’éléments dans les objets alors on utilise les combinaisons. Ainsi un tirage simultané de $p$-éléments correspond à une combinaison.
\end{itemize}
\subsection*{\underline{\textbf{\textcolor{red}{Exercice d’application}}}}
Un lycée a choisi ses 15 délégués de classe : 7 garçons et 8 filles.

\subsection*{1. Pour représenter le lycée à un jumelage, ces délégués doivent choisir entre eux une délégation de 5 membres (les membres de la délégation ne jouent aucun rôle).}
\begin{enumerate}
    \item[a.] Combien y a-t-il de délégations possibles ? \\
    
    \item[b.] Déterminer le nombre de délégations comprenant 2 garçons et 3 filles. \\
    
    \item[c.] Déterminer le nombre de délégations comprenant au moins une fille. \\
\end{enumerate}

\subsection*{2. Ces délégués se réunissent pour élire un gouvernement scolaire de 5 membres, comprenant un président, un premier ministre, un ministre de l’intérieur, un ministre de la culture et des sports et un ministre des finances, sans cumul de postes.}
\begin{enumerate}
    \item[a.] Déterminer le nombre de gouvernements possibles. \\
    
    \item[b.] Déterminer le nombre de gouvernements tels que le président est une fille. \\
    
    \item[c.] Déterminer le nombre de gouvernements tel que le ministre de l’intérieur est un garçon. \\
    
    \item[d.] Déterminer le nombre de gouvernements comprenant exactement un garçon. \\
\end{enumerate}
\section*{\underline{\textbf{\textcolor{red}II. Probabilité}}}
\subsection*{\underline{\textbf{\textcolor{red}{1. Expérience aléatoire et univers}}}}
\begin{enumerate}
\item[•]On dit qu’une expérience est aléatoire si on ne peut pas prédire avec certitude son résultat mais on peut décrire l’ensemble de tous ses résultats possibles. Une expérience aléatoire est dite épreuve.
\item[•]L’ensemble des résultats possibles d’une épreuve est appelé univers et est souvent noté $\Omega$.
\item[a.] \textbf{Exemple}

Quand on lance un dé cubique dont les faces sont

numérotées de 1 à 6 puis on note le chiffre apparu 
sur sa face supérieure, on a une épreuve dont l’univers $\Omega = \lbrace 1; 2; 3; 4; 5; 6\rbrace$
\item[b.] \textbf{Eventualité d’une épreuve}

Dans une épreuve, on appelle éventualité, tout résultat possible de l’épreuve. Par exemple dans 
l’épreuve ci-dessus $1 ; 2 ; 6\cdots $ sont des éventualités.
\item[c.] \textbf{ Événement d’une épreuve}

Dans une épreuve d’univers $\Omega$, on appelle évènement, toute partie de $\Omega$ c’est-à-dire tout ensemble d’éventualités. Par exemple dans l’épreuve ci-dessus d’univers
$\Omega= \lbrace1; 2; 3; 4; 5; 6\rbrace $ , $\lll$ obtenir un chiffre 
pair $\ggg$ $= \lbrace2; 4; 6\rbrace $ est un événement que l’on peut noter A, on a alors 
$A = \lbrace 2; 4; 6\rbrace$

\item[d.] \textbf{Remarque}

Dans une épreuve :
\item[•] L’ensemble vide est un événement appelé évènement impossible.
\item[•] L’univers $\Omega$ est un événement appelé évènement certain.
\item[•] un événement contenant une seule éventualité est appelé évènement élémentaire. Par exemple, 
les événements $\lbrace 1\rbrace$ ;$\lbrace 5\rbrace$ et $\lbrace 4\rbrace$ sont des événements élémentaires de l’épreuve ci-dessus.
\end{enumerate}
\subsection*{\underline{\textbf{\textcolor{red}{2.  Évènement$ \lll A et B \ggg $; événement$ \lll A ou B \ggg $}}}}
Soient A et B deux événements dans une épreuve.
\begin{itemize}
\item[•] L’événement $A \cap B$ est dit événement $\lll$ A et B $\ggg$ 
\item[•] Quand $A \cap B = \emptyset$, on dit que $A$ et $B$ sont incompatibles.
\item[•] L’événement $A \cup B$ est dit événement $\lll A$ ou $B \ggg$
\end{itemize}
\subsection*{\underline{\textbf{\textcolor{red}{3. Évènement contraire}}}}
Dans une épreuve d’univers $\Omega$, L’événement $\overline{A}$ est dit événement contraire de l’événement $A$, Par exemple dans 
l’épreuve ci-dessus d’univers $\Omega = \lbrace1; 2; 3; 4; 5; 6\rbrace$, l’événement contraire de 
$A = \lbrace2; 4; 6\rbrace$ est $\overline{A}= \lbrace1; 3; 5\rbrace.$
\subsection*{\underline{\textbf{\textcolor{red}{4. Notion de probabilité}}}}
\begin{enumerate}
\item[a.]\textbf{Hypothèse d’équiprobabilité}

Dans une épreuve d’univers $\Omega =\lbrace e_{1}; e_{2}; \cdots ; e_{n}\rbrace$, on dit qu’il y a\\ équiprobabilité si tous les événements élémentaires ont la même probabilité. Dans les exercices, l’équiprobabilité est suggérée par des expressions comme : Dé parfait ; boules indiscernables au toucher ; pièce équilibrée, cartes bien battues ; tirer au hasard $\cdots$
\item[b.]\textbf{Propriété}

Dans une épreuve d’univers $\Omega$, s’il y a équiprobabilité alors la probabilité d’un événement $A$ est le réel noté $P(A)$ et défini par $P(A)=\frac{card(A)}{card(\Omega)}$
\item[c.]\textbf{Remarque}
\begin{itemize}
\item[•] La probabilité de l’événement impossible est $P(\emptyset) = 0$
\item[•] La probabilité de l’événement certain est $P(\Omega)=1$
\item[•] La probabilité d’un événement $A$, $P(A)$ appartient à l’intervalle $[0; 1]$.
\end{itemize}
\item[d.]\textbf{Propriétés}
\begin{itemize}
\item[•] $P(A) =1-P(\overline{A})$ et $P(\overline{A}) =1-p(A)$
\item[•]  Si $A$ et $B$ sont des événements incompatibles alors\\ $P(A \cup B) = P(A) + P(B)$
\item[•]  Si $A$ et $B$ sont des événements quelconques alors\\ $P(A \cup B) = P(A) + P(B) - P(A \cap B)$
\end{itemize}
\end{enumerate}
\subsection*{\underline{\textbf{\textcolor{red}{Exercices d’application}}}}
\subsection*{\underline{\textbf{\textcolor{red}{Exercice 1(Bac 2018)}}}}
Une boite contient les huit lettres du mot CAFEBOOK. Un élève tire au hasard et simultanément 
trois lettres de la boite.

a. Calculer le nombre de tirages possibles.

b. Calculer la probabilité de tirer deux consonnes et une voyelle.

c. Calculer la probabilité d’obtenir au moins une voyelle.
\subsection*{\underline{\textbf{\textcolor{red}{Exercice 2(Bac 2019)}}}}
Un joueur de football dispose de trois paires de chaussures de couleurs différentes dans un tiroir. 
A quelques heures d’un match, le joueur tire au hasard deux chaussures du tiroir. Déterminer la 
probabilité des événements suivants :

A «  Elles appartiennent à la même paire ».

B «  Le joueur ne peut pas jouer avec ce tirage »

C «  il y a un pied droit et un pied gauche ».

fryuhijedzokpurfiojeiurfedkirkjguhjk

defryuhjkjhuiuy
\end{document}
