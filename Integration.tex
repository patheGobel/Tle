\documentclass[12pt]{article}
\usepackage{stmaryrd}
\usepackage{graphicx}
\usepackage[utf8]{inputenc}
\usepackage[french]{babel}
\usepackage[T1]{fontenc}
\usepackage{hyperref}
\usepackage{verbatim}
\usepackage{color, soul}
\usepackage{pgfplots}
\pgfplotsset{compat=1.15}
\usepackage{mathrsfs}
\usepackage{amsmath}
\usepackage{amsfonts}
\usepackage{amssymb}
\usepackage{tkz-tab}
\author{Destiné aux élèves de Terminale L\\Lycée de Dindéfelo\\Présenté par M. BA}
\title{\textbf{Calcul Intégral}}
\date{\today}
\usepackage{tikz}
\usetikzlibrary{arrows, shapes.geometric, fit}
\newcommand{\myul}[2][black]{\setulcolor{#1}\ul{#2}\setulcolor{black}}
\newcommand\tab[1][1cm]{\hspace*{#1}}

\begin{document}
\maketitle
\newpage

\section*{\underline{\textbf{\textcolor{red}{I. Primitive}}}}

\subsection*{\underline{\textbf{\textcolor{red}{1. Définition}}}}
Une fonction $F$ est dite être une primitive d'une fonction $f$ sur un intervalle $I$ si $F$ est dérivable sur $I$ et si pour tout $x$ dans $I$, $F'(x) = f(x)$.

\subsection*{\underline{\textbf{\textcolor{red}{Exemple}}}}
Trouver une primitive de la fonction $f(x) = 3x^2$.

\subsection*{\underline{\textbf{\textcolor{green}{Solution}}}}
Une primitive de $f$ est la fonction $F(x) = x^3 + C$, où $C$ est une constante.

\subsection*{\underline{\textbf{\textcolor{red}{2. Propriétés}}}}
Une Fonction admet une infinité de prmitive.

Autrement dit: Les primitives d'une même fonction diffèrent d'une constante.
\subsection*{\underline{\textbf{\textcolor{red}{3. Primitive des fonctions usuelles}}}}
\begin{table}[h]
\centering
\begin{tabular}{|c|c|}
\hline
fonction \(f\) & Primitive \(F\) \\ \hline
\(a\) & \(ax + C\) \\ \hline
\(ax^{n}\) & \(\frac{ax^{n+1}}{n+1} + C\) \\ \hline
\(\frac{1}{x^{n}}\) & \(\frac{-1}{(n-1)x^{n-1}} + C\) \\ \hline
\(\frac{1}{\sqrt{x}}\) & \(2\sqrt{x} + C\) \\ \hline
\(e^{ax}\) & \(\frac{1}{a}e^{ax} + C\) \\ \hline
\(\frac{u'}{u}\) & \(\ln |u| + C\) \\ \hline
\end{tabular}
\end{table}

\subsection*{\underline{\textbf{\textcolor{red}{Exemple}}}}
Calculer $\int_{0}^{2} 4x^3 \,dx$.

\subsection*{\underline{\textbf{\textcolor{green}{Solution}}}}
La primitive de $4x^3$ est $F(x) = x^4 + C$. Ainsi, 
\[
\int_{0}^{2} 4x^3 \,dx = [x^4]_{0}^{2} = 2^4 - 0^4 = 16.
\]
\section*{\underline{\textbf{\textcolor{red}{II.Integrale des fonctions}}}}
\subsection*{\underline{\textbf{\textcolor{red}{Défintiton et propriété}}}}
Soit $f$ une fonction continue sur un intervalle $I$, on appelle intégrale de $f$ prise entre a et b le réel noté:

$\int_{a}^{b}f(x)dx=F(b)-F(a)$

\definecolor{qqwuqq}{rgb}{0,0.39215686274509803,0}
\begin{tikzpicture}[line cap=round,line join=round,>=triangle 45,x=1cm,y=1cm]
\begin{axis}[
x=1cm,y=1cm,
axis lines=middle,
ymajorgrids=true,
xmajorgrids=true,
xmin=-3.217293613624258,
xmax=4.076023470870133,
ymin=-7.615190605118021,
ymax=6.64985001698064,
xtick={-3,-2,...,25},
ytick={-7,-6,...,6},]
\clip(-3.217293613624258,-7.615190605118021) rectangle (25.076023470870133,6.64985001698064);
\draw[line width=2pt,color=qqwuqq,smooth,samples=100,domain=-3.217293613624258:25.076023470870133] plot(\x,{sin(((\x))*180/pi)});
%\draw [line width=2pt,color=ffqqqq] (-0.3,0.2)-- (-0.4,0);
%\draw [line width=2pt,color=ffqqqq] (-1.2,0.6)-- (-1.2,0);
\begin{scriptsize}
\draw[color=qqwuqq] (-3.,0.11) node {$C_{f}$};
%\draw[color=ffqqqq] (0.5,-0.09) node {$b$};
%\draw[color=ffqqqq] (-0.5,0.1) node {$a$};
\end{scriptsize}
\end{axis}
\end{tikzpicture}
\subsection*{\underline{\textbf{\textcolor{red}{Exemple}}}}
Calculer:
\[\int_{1}^{2}x^{2}+1dx\] et \[\int_{2}^{4}\frac{2x}{x^{2}-3} dx\]
\subsection*{\underline{\textbf{\textcolor{green}{Solution}}}}
\subsection*{\underline{\textbf{\textcolor{red}{Remarque}}}}
La valeur absolue de l'intégrale d'une fonction $f$ est l'aire de la surface délimité par l'axe $(x'x)$ de la courbe representative de $f$ et les droites d'équations $x=a$ et $x=b$

\definecolor{qqwuqq}{rgb}{0,0.39215686274509803,0}
\begin{tikzpicture}[line cap=round,line join=round,>=triangle 45,x=1cm,y=1cm]
\begin{axis}[
x=1cm,y=1cm,
axis lines=middle,
ymajorgrids=true,
xmajorgrids=true,
xmin=-3.217293613624258,
xmax=4.076023470870133,
ymin=-7.615190605118021,
ymax=6.64985001698064,
xtick={-3,-2,...,25},
ytick={-7,-6,...,6},]
\clip(-3.217293613624258,-7.615190605118021) rectangle (25.076023470870133,6.64985001698064);
\draw[line width=2pt,color=qqwuqq,smooth,samples=100,domain=-3.217293613624258:25.076023470870133] plot(\x,{sin(((\x))*180/pi)});
%\draw [line width=2pt,color=ffqqqq] (-0.3,0.2)-- (-0.4,0);
%\draw [line width=2pt,color=ffqqqq] (-1.2,0.6)-- (-1.2,0);
\begin{scriptsize}
\draw[color=qqwuqq] (-3.,0.11) node {$C_{f}$};
%\draw[color=ffqqqq] (0.5,-0.09) node {$b$};
%\draw[color=ffqqqq] (-0.5,0.1) node {$a$};
\end{scriptsize}
\end{axis}
\end{tikzpicture}
\\
$\mathrm{A}ire=\mid \int_{a}^{b}f(x) dx \mid Ua$

$\mid F(b)-F(a)\mid Ua$ avec $Ua$ unité d'aire

\subsection*{\underline{\textbf{\textcolor{red}{3. Exercice d’application (Bac 2018)}}}}
Soit la fonction numérique h définie par $h(x) =\frac{2e^{x}+1}{e^{x}+1}$

a.Montrer que le domaine de définition Dh de h est $\mathbb{R}$.

b. Montrer que pour tout x $\in$ Dh , $h(x) = 1 + \frac{e^{x}}{e^{x}+1}$

c.Soit la fonction k définie sur $\mathbb{R}$ par $k(x)=x+\ln(e^{x}+1)$

Montrer que k est une primitive de h sur $\mathbb{R}$.

d. Calculer l’intégrale $\int_{0}^{2}h(x)dx$
\subsection*{\underline{\textbf{\textcolor{red}{3. Exercice d’application (Bac 2018)}}}}

\begin{itemize}
    \item[a.] Soit la fonction numérique \( h \) définie par \( h(x) =\frac{2e^{x}+1}{e^{x}+1} \). Montrons que le domaine de définition \( D_h \) de \( h \) est \( \mathbb{R} \).
    
    \item[b.] Montrons que pour tout \( x \in D_h \), \( h(x) = 1 + \frac{e^{x}}{e^{x}+1} \).
    
    \item[c.] Soit la fonction \( k \) définie sur \( \mathbb{R} \) par \( k(x)=x+\ln(e^{x}+1) \). Montrons que \( k \) est une primitive de \( h \) sur \( \mathbb{R} \).
    
    \item[d.] Calculons l’intégrale \( \int_{0}^{2}h(x)dx \).
\end{itemize}
\subsection*{\underline{\textbf{\textcolor{green}{Coreection De L'Exercice d’application (Bac 2018)}}}}
\end{document}