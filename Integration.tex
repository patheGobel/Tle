\documentclass[12pt]{article}
\usepackage{stmaryrd}
\usepackage{graphicx}
\usepackage[utf8]{inputenc}
\usepackage[french]{babel}
\usepackage[T1]{fontenc}
\usepackage{hyperref}
\usepackage{verbatim}
\usepackage{color, soul}
\usepackage{pgfplots}
\pgfplotsset{compat=1.15}
\usepackage{mathrsfs}
\usepackage{amsmath}
\usepackage{amsfonts}
\usepackage{amssymb}
\usepackage{tkz-tab}
\author{Destinés à la TLe\\Au Lycée de Dindéferlo}
\title{\textbf{Calcul Intégral}}
\date{\today}
\usepackage{tikz}
\usetikzlibrary{arrows, shapes.geometric, fit}
\newcommand{\myul}[2][black]{\setulcolor{#1}\ul{#2}\setulcolor{black}}
\newcommand\tab[1][1cm]{\hspace*{#1}}

\begin{document}
\maketitle
\newpage

\section*{\underline{\textbf{\textcolor{red}{I. Primitive}}}}

\subsection*{\underline{\textbf{\textcolor{red}{1. Définition}}}}
Une fonction $F$ est dite être une primitive d'une fonction $f$ sur un intervalle $I$ si $F$ est dérivable sur $I$ et si pour tout $x$ dans $I$, $F'(x) = f(x)$.

\subsection*{\underline{\textbf{\textcolor{red}{Exemple}}}}
Trouver une primitive de la fonction $f(x) = 3x^2$.

\subsection*{\underline{\textbf{\textcolor{red}{Solution}}}}
Une primitive de $f$ est la fonction $F(x) = x^3 + C$, où $C$ est une constante.

\subsection*{\underline{\textbf{\textcolor{red}{2. Propriétés}}}}
Une Fonction admet une infinité de prmitive.

Autrement dit: Les primitives d'une même fonction diffèrent d'une constante.
\subsection*{\underline{\textbf{\textcolor{red}{3. Primitive des fonctions usuelles}}}}
\begin{table}[h]
\centering
\begin{tabular}{|c|c|}
\hline
fonction \(f\) & Primitive \(F\) \\ \hline
\(a\) & \(ax + C\) \\ \hline
\(ax^{n}\) & \(\frac{ax^{n+1}}{n+1} + C\) \\ \hline
\(\frac{1}{x^{n}}\) & \(\frac{-1}{(n-1)x^{n-1}} + C\) \\ \hline
\(\frac{1}{\sqrt{x}}\) & \(2\sqrt{x} + C\) \\ \hline
\(e^{ax}\) & \(\frac{1}{a}e^{ax} + C\) \\ \hline
\(\frac{u'}{u}\) & \(\ln |u| + C\) \\ \hline
\end{tabular}
\end{table}

\subsection*{\underline{\textbf{\textcolor{red}{Exemple}}}}
Calculer $\int_{0}^{2} 4x^3 \,dx$.

\subsection*{\underline{\textbf{\textcolor{red}{Solution}}}}
La primitive de $4x^3$ est $F(x) = x^4 + C$. Ainsi, 
\[
\int_{0}^{2} 4x^3 \,dx = [x^4]_{0}^{2} = 2^4 - 0^4 = 16.
\]
\section*{\underline{\textbf{\textcolor{red}{II.Integrale des fonctions}}}}
\subsection*{\underline{\textbf{\textcolor{red}{Défintiton et propriété}}}}
Soit $f$ une fonction continue sur un intervalle $I$, on appelle intégrale de $f$ prise entre a et b le réel noté:

$\int_{a}^{b}f(x)dx=F(b)-F(a)$

\definecolor{ffqqqq}{rgb}{1,0,0}
\definecolor{qqwuqq}{rgb}{0,0.39215686274509803,0}
\begin{tikzpicture}[line cap=round,line join=round,>=triangle 45,x=1cm,y=1cm]
\begin{axis}[
x=1cm,y=1cm,
axis lines=middle,
ymajorgrids=true,
xmajorgrids=true,
xmin=-2.243449863788017,
xmax=2.579354548354068,
ymin=-1.2487544863289648,
ymax=1.7859605615202274,
xtick={-2.2,-2,...,2.4},
ytick={-1.2000000000000002,-1.0000000000000002,...,1.6},]
\clip(-2.243449863788017,-1.2487544863289648) rectangle (2.579354548354068,1.7859605615202274);
\draw[line width=2pt,color=qqwuqq,smooth,samples=100,domain=-2.243449863788017:2.579354548354068] plot(\x,{(\x)^(2)-(\x)^(3)+(\x)^(5)});
\draw [line width=2pt,color=ffqqqq] (-0.3932074994028332,0.20600719848518462)-- (-0.4,0);
\draw [line width=2pt,color=ffqqqq] (-1.2017951223709928,0.6730904551654109)-- (-1.2,0);
\begin{scriptsize}
\draw[color=qqwuqq] (-1.4375504319940398,-1.187367615547783) node {$f$};
\draw[color=ffqqqq] (-0.39554765104167,-0.09499612010829164) node {$b$};
\draw[color=ffqqqq] (-1.107005743172291,0.1599954969827712) node {$a$};
\end{scriptsize}
\end{axis}
\end{tikzpicture}
\subsection*{\underline{\textbf{\textcolor{red}{Exemple}}}}
Calculer:
\[\int_{1}^{2}x^{2}+1dx\] et \[\int_{2}^{4}\frac{2x}{x^{2}-3} dx\]
\subsection*{\underline{\textbf{\textcolor{red}{Solution}}}}
\subsection*{\underline{\textbf{\textcolor{red}{Remarque}}}}
La valeur absolue de l'intégrale d'une fonction $f$ est l'aire de la surface délimité par l'axe $(x'x)$ de la courbe representative de $f$ et les droites d'équations $x=a$ et $x=b$

\end{document}
