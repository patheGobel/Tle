\documentclass[12pt]{article}
\usepackage{stmaryrd}
\usepackage{graphicx}
\usepackage[utf8]{inputenc}

\usepackage[french]{babel}
\usepackage[T1]{fontenc}
\usepackage{hyperref}
\usepackage{verbatim}

\usepackage{color, soul}

\usepackage{pgfplots}
\pgfplotsset{compat=1.15}
\usepackage{mathrsfs}

\usepackage{amsmath}
\usepackage{amsfonts}
\usepackage{amssymb}
\usepackage{tkz-tab}
\author{Destinés à la TLe\\Au Lycée de Dindéferlo}
\title{\textbf{Suite Numérique }}
\date{\today}
\usepackage{tikz}
\usetikzlibrary{arrows, shapes.geometric, fit}

% Commande pour la couleur d'accentuation
\newcommand{\myul}[2][black]{\setulcolor{#1}\ul{#2}\setulcolor{black}}
\newcommand\tab[1][1cm]{\hspace*{#1}}

\begin{document}
\maketitle
\newpage
\section*{\underline{\textbf{\textcolor{red}{I. Généralités}}}}
\subsection*{\underline{\textbf{\textcolor{red}{1. Définition}}}}
Une suite est une fonction $u\ :\ \mathbb{N}\longrightarrow\mathbb{R}$ définie sur l'ensemble $\mathbb{N}$ des entiers naturels.
	
L'image par la suite $u$ de l'entier $n$ est notée $u_{n}$ au lieu de $u(n)$
	
La suite elle-même est notée $\left(u_{n}\right)$
\subsection*{\underline{\textbf{\textcolor{red}{2. Modes de définition d'une suite }}}}
$\bullet\ $On peut définir une suite par une formule de la forme : $u_{n}=f(n)$ (définition explique).
\subsection*{\underline{\textbf{\textcolor{red}{Exemple :}}}}
Soit la suite $\left(u_{n}\right)$ définie par : $u_{n}=n^{2}-5n+3.$

Alors :
 
$u_{0}=0^{2}-(5\times 0)+3=3$ ;
	
$u_{1}=1^{2}-(5\times 1)+3=-1$ ;
	
$u_{2}=2^{2}-(5\times 2)+3=-3$ ;
	
$u_{10}=10^{2}-(5\times 10)+3=53$ ;
	
$u_{50}=50^{2}-(5\times 50)+3=2253$
	
$\bullet\ $On peut aussi définir une suite par une condition de la forme : $u_{n+1}=f\left(u_{n}\right)$ et la donnée du premier terme $u_{0}$ (relation de récurrence).
	
$\bullet\ $On peut alors calculer de proche en proche les termes de la suite : $u_{1}=f\left(u_{0}\right)\ ;\ u_{2}=f\left(u_{1}\right)\ ;\ u_{3}=f\left(u_{2}\right)\ ;\ \text{ etc}\ldots$
\subsection*{\underline{\textbf{\textcolor{red}{Exemple :}}}}
	
Soit la suite $\left(u_{n}\right)$ définie par :
	
$u_{n+1}=2u_{n}+3\text{ et }u_{0}=-1.$
	
Alors : $u_{1}=u_{0+1}=2u_{0}+3=1$ ;
	
$u_{2}=u_{1+1}=2u_{1}+3=-3$ ;
	
$u_{3}=u_{2+1}=2u_{2}+3=13\ ;\ \text{ etc}\ldots$
	
Par exemple, pour calculer $u_{50}$, il faudrait faire $50$ calculs successifs.
\subsection*{\underline{\textbf{\textcolor{red}{3. Sens de variation d'une suite}}}}
\subsection*{\underline{\textbf{\textcolor{red}{Définition :}}}}
	
Une suite $\left(u_{n}\right)$ est dite :
	
$-\ $Croissante si : $\forall\;n\;,\ :\ u_{n+1}\geq u_{n}.$
	
$-\ $Décroissante si : $\forall\;n;,\ :\ u_{n+1}\leq u_{n}$

$-\ $Monotone si elle est croissante ou décroissante.

$-\ $Constante si : $\forall\;n\;,\ :\ u_{n+1}=u_{n}.$

Étudier le sens de variation d'une suite $\left(u_{n}\right)$

C'est dire si elle est croissante ou décroissante ou constante.
\subsection*{\underline{\textbf{\textcolor{red}{Règle :}}}}
Pour étudier le sens de variation d'une suite $\left(u_{n}\right)$,on compare deux termes consécutifs, pour cela, on peut étudier le signe de leur différence, ou, s'il s'agit de nombres strictement positifs, comparer leur quotient à $1.$
\subsection*{\underline{\textbf{\textcolor{red}{Exemple :}}}}
Soit la suite $\left(u_{n}\right)$ définie par : $u_{n}=\dfrac{n+2}{2n+1}$

Alors :

$u_{n+1}=\dfrac{(n+1)+2}{2(n+1)+1}=\dfrac{n+3}{2n+3}$
	
$u_{n+1}-u_{n}=\dfrac{n+3}{2n+3}-\dfrac{n+2}{2n+1}=\dfrac{-3}{(2n+1)(2n+3})$
	
Pour tout entier naturel $n$, on a donc : $u_{n+1}-u_{n}\leq 0.$
	
La suite étudiée est par conséquent décroissante.
\section*{\underline{\textbf{\textcolor{red}{II. Suite arithmétiques}}}}
Une suite $\left(u_{n}\right)$ est arithmétique si chaque terme s'obtient en ajoutant au précédent un même nombre $r$ appelé raison : $u_{n+1}=u_{n}+r.$
	 
\subsection*{\underline{\textbf{\textcolor{red}{1.Expression du terme général}}}}		
$\bullet\ \text{Si }\left(u_{n}\right)$ est une suite arithmétique de premier terme $u_{0}$ et de raison $r$, alors :
	
$$u_{n}=u_{0}+nr$$
	
$\bullet\ $Si le premier terme est $u_{1}$, alors :
$$u_{n}=u_{1}+(n-1)r$$
\subsection*{\underline{\textbf{\textcolor{red}{2.Somme des premiers termes}}}}	
$\bullet\ $Si la suite a pour premier terme $u_{0}$, alors la somme $S_{n}=u_{0}+u_{1}+\ldots+u_{n}$ vaut :
	
$$S_{n}=\dfrac{(n+1)\left(u_{0}+u_{n}\right)}{2}$$

$\bullet\ $Si la suite a pour premier terme $u_{1}$, alors la somme $S_{n}=u_{1}+u_{1}+\ldots+u_{n}$ vaut :

$$S_{n}=\dfrac{n(u_{1}+u_{n})}{2}$$

\subsection*{\underline{\textbf{\textcolor{red}{Exemple :}}}}
Soit $\left(u_{n}\right)$ la suite définie par : $u_{n}=2n-1$ et $u_{1}=1.$

Alors : $\left(u_{n}\right)$ est une suite arithmétique

Car : $u_{n+1}-u_{n}=2(n+1)-1-(2n-1)=2n+2-1-2n+1=2.$

Donc : $u_{n+1}=u_{n}+2.$ 

La raison de la suite est $2.$

La somme des $n$ premiers termes vaut : $u_{1}+u_{2}\ldots+u_{n}=\dfrac{n(1+2n-1}{2}=n^{2}$
\section*{\underline{\textbf{\textcolor{red}{III. Suites géométriques}}}}
Une suite $\left(u_{n}\right)$ est dite géométrie si chaque terme s'obtient en multipliant le précédent par un même nombre $q$ appelé raison : $u_{n+1}=u_{n}\times q.$
\subsection*{\underline{\textbf{\textcolor{red}{1. Expression du terme général}}}}
$\bullet\ $Si la suite géométrique $\left(u_{n}\right)$ a pour premier terme $u_{0}$ et pour raison $q$, alors : $$u_{n}=u_{0}\times q^{n}$$

$\bullet\ $Si le premier terme est $u_{1}$, alors :$$u_{n}=u_{1}\times q^{n-1}$$
\subsection*{\underline{\textbf{\textcolor{red}{2. Somme des premiers termes }}}}

Pour toute suite géométrique, de raison $q\neq 1$, on a:

$$u_{0}+u_{1}+\ldots+u_{n}=u_{0}\times\dfrac{1-q^{n+1}}{1-q}$$

$$u_{1}+u_{2}+\ldots+u_{n}=u_{1}\times\dfrac{1-q^{n}}{1-q}$$

\subsection*{\underline{\textbf{\textcolor{red}{Exemple :}}}}
$1+\dfrac{1}{2}+\dfrac{1}{2^{2}}+\ldots\ldots+\dfrac{1}{2^{n}}=S_{n}$ est la somme des premiers termes d'une suite géométrique de raison $\dfrac{1}{2}.$ 

Donc : $1+\dfrac{1}{2}+\dfrac{1}{2^{2}}+\ldots\ldots+\dfrac{1}{2^{n}}=\dfrac{1-\left(\dfrac{1}{2}\right)^{n+1}}{1-\dfrac{1}{2}}=2-\dfrac{1}{2^{n}}$
\section*{\underline{\textbf{\textcolor{red}{IV. limites d'une suite}}}}
La notion de limite en $+\infty$, déjà rencontrée à propos des fonction, s'étend au cas des suites.
	
On a les résultats suivantes :
\subsection*{\underline{\textbf{\textcolor{red}{Théorème 1 :}}}}
a. $\lim\limits_{n\;\longrightarrow\;+\infty}\sqrt{n}=+\infty\ ;\ \lim\limits_{n\;\longrightarrow\;+\infty}n^{2}=+\infty\ ;\ \lim\limits_{n\;\longrightarrow\;+\infty}n^{3}=+\infty.$

b. $\lim\limits_{n\;\longrightarrow\;+\infty}\dfrac{1}{\sqrt{n}}=0\ ;\ \lim\limits_{n\;\longrightarrow\;+\infty}\dfrac{1}{n}=0\ ;\ \lim\limits_{n\;\longrightarrow\;+\infty}\dfrac{1}{n^{2}}=0\ ;\ \lim\limits_{n\;\longrightarrow\;+\infty}\dfrac{1}{n^{3}}=0.$
\subsection*{\underline{\textbf{\textcolor{red}{Théorème 2 :}}}}
Soit $q$ un nombre réel.
Soit \( q \) un nombre réel.

\begin{itemize}
    \item Si \( q > 1 \), alors
    \[
    \lim_{n \rightarrow +\infty} q^n = +\infty.
    \]
    
    \item Si \( -1 < q < 1 \), alors
    \[
    \lim_{n \rightarrow +\infty} q^n = 0.
    \]
\end{itemize}
\subsection*{\underline{\textbf{\textcolor{red}{Théorème 3 :}}}}
Les résultats concernant les opérations sur les limites de fonctions s'étendent aux limites de suites.
\subsection*{\underline{\textbf{\textcolor{red}{Exemple :}}}}

1) Soit la suite \( (u_{n}) \) définie par : \( u_{n} = \dfrac{3n^{3} - 5n^{2} + 1}{2n^{3} + 1} \)

Alors \( \lim_{n \rightarrow +\infty} u_{n} = \lim_{n \rightarrow +\infty} \dfrac{3n^{3}}{2n^{3}} = \dfrac{3}{2} \).

2) Soit la suite \( (v_{n}) \) définie par : \( v_{n} = 1 + \dfrac{1}{3} + \left(\dfrac{1}{3}\right)^{2} + \ldots + \left(\dfrac{1}{3}\right)^{n+1} \).

On a d'après le paragraphe III :

\[
v_{n} = \dfrac{1 - \left(\dfrac{1}{3}\right)^{''}}{1 - \dfrac{1}{3}}
\]

car \( v_{n} \) est la somme des termes consécutifs d'une suite géométrique de raison \( \dfrac{1}{3} \), or comme \( -1 < \dfrac{1}{3} < 1 \), \( \lim_{n \rightarrow +\infty} \left(\dfrac{1}{3}\right)^{''} = 0 \).

D'où : \( \lim_{n \rightarrow +\infty} \dfrac{1}{1 - \dfrac{1}{3}} = \dfrac{3}{2} \).
\end{document}