\documentclass[12pt]{article}
\usepackage{stmaryrd}
\usepackage{graphicx}
\usepackage[utf8]{inputenc}

\usepackage[french]{babel}
\usepackage[T1]{fontenc}
\usepackage{hyperref}
\usepackage{verbatim}

\usepackage{color, soul}

\usepackage{pgfplots}
\pgfplotsset{compat=1.15}
\usepackage{mathrsfs}

\usepackage{amsmath}
\usepackage{amsfonts}
\usepackage{amssymb}
\usepackage{tkz-tab}
\author{}
\title{\textbf{SUITES NUMERIQUES}}
\date{\today}
\usepackage{tikz}
\usetikzlibrary{arrows, shapes.geometric, fit}

% Commande pour la couleur d'accentuation
\newcommand{\myul}[2][black]{\setulcolor{#1}\ul{#2}\setulcolor{black}}
\newcommand\tab[1][1cm]{\hspace*{#1}}

\begin{document}
\maketitle
\newpage

\section*{\underline{\textbf{\textcolor{red}{I. Notion de suite}}}} 
\subsection*{\underline{\textbf{\textcolor{red}{1. Activité}}}} 
On considère les nombrés réels suivants : -10,1; 5; 3,4 et - 8

1. Ranger cette liste de 4 nombres dans l’ordre décroissant.

2. En considérant que le plus grand nombre occupe le rang 0 et en prenant l’ordre décroissant, donner les
rangs des autres nombres.

\underline{\textbf{\textcolor{red}{1.Solution}}}
\begin{itemize}
\item[1.]Le rangement dans l’ordre décroissant de cette liste de 4 nombres est : 5 ; 3,4 ; -8 ; -10,1.

\item[2.] 15,75 ; 15,25 ; 15 ; 14,75
\end{itemize}
\underline{\textbf{\textcolor{red}{Exploitation de l'activité}}}

La liste ordonnée suivant l’ordre décroissant des nombres suivants: 5; 3,4; -8; -10,1 est un exemple de suite de
nombres réels. Cette suite peut être appelée u. Ainsi on peut écrire u : 5; 3,4; -8; -10,1 . Chaque élément de cette suite est dit terme de la suite u et peut être repéré par son rang. Ainsi, -8 est un terme de la suite u. Pour noter un terme d’une suite, on écrit le nom de la suite et on met en indice son rang, ainsi pour la suite u ci-dessus, on a : $u_{0}=5$ ; $u_{1}=3,4$ ; $u_{2}=-8$ et $u_{3}=-10,1$
\section*{\underline{\textbf{\textcolor{red}{2. Définition, vocabulaire et notation}}}}
Une suite numérique est une liste ordonnée de nombres réels. Chaque élément d’une suite est appelé terme
et peut être repéré dans la liste par son rang. Pour une suite numérique u, le terme de rang n noté un (n étant
une variable qui est un entier naturel) est dit terme général.

L’ensemble des rangs (indices) des termes d’une suite numérique est\\
$\mathbb{N}=\left\lbrace 0 ; 1 ; 2 ; ... \right\rbrace$ ou $\mathbb{N}^{*}=\left\lbrace 1 ; 2 ; ... \right\rbrace $

Pour noter une suite numérique u alors on met le terme général $u_{n}$ entre parenthèses puis on précise
l’ensemble des indices de ses termes. Par exemple on peut avoir $(u_{n})_{n\in\mathbb{N}^{*}}\cdots$ 
\section*{\underline{\textbf{\textcolor{red}{II. Suites arithmétiques}}}}
\section*{\underline{\textbf{\textcolor{red}{III. Suites géométriques}}}}
\end{document}
