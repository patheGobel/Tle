\documentclass[12pt]{article}
\usepackage{stmaryrd}
\usepackage{graphicx}
\usepackage[utf8]{inputenc}

\usepackage[french]{babel}
\usepackage[T1]{fontenc}
\usepackage{hyperref}
\usepackage{verbatim}

\usepackage{color, soul}

\usepackage{pgfplots}
\pgfplotsset{compat=1.15}
\usepackage{mathrsfs}

\usepackage{amsmath}
\usepackage{amsfonts}
\usepackage{amssymb}
\usepackage{tkz-tab}
\author{}
\title{\textbf{SUITES NUMERIQUES}}
\date{\today}
\usepackage{tikz}
\usetikzlibrary{arrows, shapes.geometric, fit}

% Commande pour la couleur d'accentuation
\newcommand{\myul}[2][black]{\setulcolor{#1}\ul{#2}\setulcolor{black}}
\newcommand\tab[1][1cm]{\hspace*{#1}}

\begin{document}
\maketitle
\newpage

\section*{\underline{\textbf{\textcolor{red}{I. Notion de suite}}}} 
\subsection*{\underline{\textbf{\textcolor{red}{1. Activité}}}} 
On considère les nombrés réels suivants : -10,1; 5; 3,4 et - 8

1. Ranger cette liste de 4 nombres dans l’ordre décroissant.

2. En considérant que le plus grand nombre occupe le rang 0 et en prenant l’ordre décroissant, donner les
rangs des autres nombres.

\underline{\textbf{\textcolor{red}{1.Solution}}}
\begin{itemize}
\item[1.]Le rangement dans l’ordre décroissant de cette liste de 4 nombres est : 5 ; 3,4 ; -8 ; -10,1.

\item[2.] 15,75 ; 15,25 ; 15 ; 14,75
\end{itemize}
\underline{\textbf{\textcolor{red}{Exploitation de l'activité}}}

La liste ordonnée suivant l’ordre décroissant des nombres suivants: 5; 3,4; -8; -10,1 est un exemple de suite de
nombres réels. Cette suite peut être appelée u. Ainsi on peut écrire u : 5; 3,4; -8; -10,1 . Chaque élément de cette suite est dit terme de la suite u et peut être repéré par son rang. Ainsi, -8 est un terme de la suite u. Pour noter un terme d’une suite, on écrit le nom de la suite et on met en indice son rang, ainsi pour la suite u ci-dessus, on a : $u_{0}=5$ ; $u_{1}=3,4$ ; $u_{2}=-8$ et $u_{3}=-10,1$
\section*{\underline{\textbf{\textcolor{red}{2. Définition, vocabulaire et notation}}}}
Une suite numérique est une liste ordonnée de nombres réels. Chaque élément d’une suite est appelé terme
et peut être repéré dans la liste par son rang. Pour une suite numérique u, le terme de rang n noté un (n étant
une variable qui est un entier naturel) est dit terme général.

L’ensemble des rangs (indices) des termes d’une suite numérique est\\
$\mathbb{N}=\left\lbrace 0 ; 1 ; 2 ; ... \right\rbrace$ ou $\mathbb{N}^{*}=\left\lbrace 1 ; 2 ; ... \right\rbrace $

Pour noter une suite numérique u alors on met le terme général $u_{n}$ entre parenthèses puis on précise
l’ensemble des indices de ses termes. Par exemple on peut avoir $(u_{n})_{n\in\mathbb{N}^{*}}\cdots$ 

\underline{\textbf{\textcolor{red}{Remarques}}}
\begin{itemize}
\item[•] Lorsqu’on écrit $(u_{n})_{n\in\mathbb{N}}$ alors le premier terme est $u_{0}$, le second est $u_{1}\cdots$
\item[•] Lorsqu’on écrit $(u_{n})_{n\in\mathbb{N}^{*}}$ alors le premier terme est $u_{1}$, le second est $u_{2}\cdots$
\end{itemize}
\subsection*{\underline{\textbf{\textcolor{red}{3. Exemples de suites}}}}
\subsection*{\underline{\textbf{\textcolor{red}{a. Suites définies par une formule explicite :}}}}
Une suite numérique $(u_{n})_{n\in\mathbb{N}}$ est dite définie par une formule explicite s’il existe une formule qui donne le terme général $(u_{n})$ un directement en fonction de n.

Pour une telle suite, on peut déterminer la valeur d’un terme quelconque si son indice est donné.

\underline{\textbf{\textcolor{red}{Exemple}}}

Soit $(u_{n})_{n\in\mathbb{N}}$ la suite numérique définie par $u_{n}=2n-3$

La formule $u_{n}=2n-3$ donne le terme général $u_{n}$ directement en fonction de n donc la suite $(u_{n})$ est définie par une formule explicite.

Le terme d’indice 0 est $u_{0}=-3$ ; le terme d’indice 1 est $u_{1}=-1$ et le terme de d’indice 10 est $u_{10}=17$ et le terme d’indice n+1 est\\ $u_{n+1}=2(n+1)-3=2n-1$
\subsection*{\underline{\textbf{\textcolor{red}{Exercice d'application}}}}
Soit $(u_{n})_{n\in\mathbb{N}}$ la suite numérique définie par $u_{n}=n^{2}-2$
\begin{itemize}
\item[1]. $u_{0}$; $u_{1}$ ; $u_{2}$ ; $u_{5}$ ; $u_{8}$
\item[2]. Calculer $u_{n+1}$ et $u_{n-1}$ en fonction de n.
\end{itemize}
\subsection*{\underline{\textbf{\textcolor{red}{b. Suites définies par une formule de récurrence}}}}
Une suite numérique $(u_{n})_{n\in\mathbb{N}}$ est dite définie par une formule de récurrence si la valeur de son $1^{er}$ terme
est donnée et qu’il existe une formule entre les termes généraux $u_{n+1}$ ou un ou $u_{n-1}\cdots$
Pour une telle suite, la valeur d’un terme ne peut pas être déterminée si on ne connait pas la valeur de
tous les termes qui le précèdent.

\underline{\textbf{\textcolor{red}{Exemple}}}

\[\text{Soit} (u_{n})_{n\in\mathbb{N}} \text{ la suite numérique définie par :} 
\begin{cases}
u_{0} = -3 \\
u_{n+1} =u_{n}-3
\end{cases}
\]
La valeur du $1^{er}$ terme de cette suite est $u_{0} = -3$ et la formule $u_{n+1} = 2u_{n} + 5$ est entre les termes généraux $u_{n+1}$ et $u_{n}$ . Cette formule est dite relation de récurrence et la suite est dite définie par une formule de récurrence.
\begin{itemize}
\item Pour calculer $u_{1}$ , on remplace n par 0, dans la relation de récurrence et on a : $u_{1} = 2u_{0} + 5 = -1.$

\item Pour calculer $u_{2}$ , on remplace n par 1 dans la relation de récurrence et on a : $u_{2} = 2u_{1} + 5 = 3.$

\item Pour calculer $u_{3}$ , on remplace n par 2 dans la relation de récurrence et on a : $u_{3}= 2u_{2} + 5 = 11$
\end{itemize}
\subsection*{\underline{\textbf{\textcolor{red}{Exercice d'application}}}}
\[\text{Soit} (u_{n})_{n\in\mathbb{N}} \text{ la suite numérique définie par :} 
\begin{cases}
u_{0} = -1 \\
u_{n+1} =u_{n}-3
\end{cases}
\]
Calculer les 4 premiers termes de la suite.
\subsection*{\underline{\textbf{\textcolor{red}{4. Monotonie ou sens de variation d’une suite}}}}
\section*{\underline{\textbf{\textcolor{red}{II. Suites arithmétiques}}}}
\section*{\underline{\textbf{\textcolor{red}{III. Suites géométriques}}}}
\end{document}
