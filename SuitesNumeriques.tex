\documentclass[12pt]{article}
\usepackage{stmaryrd}
\usepackage{graphicx}
\usepackage[utf8]{inputenc}

\usepackage[french]{babel}
\usepackage[T1]{fontenc}
\usepackage{hyperref}
\usepackage{verbatim}

\usepackage{color, soul}

\usepackage{pgfplots}
\pgfplotsset{compat=1.15}
\usepackage{mathrsfs}

\usepackage{amsmath}
\usepackage{amsfonts}
\usepackage{amssymb}
\usepackage{tkz-tab}
\author{Destinés à la Tle\\Au Lycée de Dindéfelo}
\title{\textbf{SUITES NUMERIQUES}}
\date{\today}
\usepackage{tikz}
\usetikzlibrary{arrows, shapes.geometric, fit}

% Commande pour la couleur d'accentuation
\newcommand{\myul}[2][black]{\setulcolor{#1}\ul{#2}\setulcolor{black}}
\newcommand\tab[1][1cm]{\hspace*{#1}}

\begin{document}
\maketitle
\newpage
\section*{\underline{\textbf{\textcolor{red}{I. Notion de suite}}}}
\subsection*{\underline{\textbf{\textcolor{red}{1. Activité}}}}
\subsection*{\underline{\textbf{\textcolor{red}{Solution}}}}
\subsection*{\underline{\textbf{\textcolor{red}{Exploitation}}}}
\subsection*{\underline{\textbf{\textcolor{red}{2. Définition, vocabulaire et notation}}}}
\subsection*{\underline{\textbf{\textcolor{red}{Remarques}}}}
\subsection*{\underline{\textbf{\textcolor{red}{3.Exemples de suites}}}}
\subsection*{\underline{\textbf{\textcolor{red}{a. Suites définies par une formule explicite :}}}}
\subsection*{\underline{\textbf{\textcolor{red}{Exemple :}}}}
\subsection*{\underline{\textbf{\textcolor{red}{Exercice d’application}}}}
\subsection*{\underline{\textbf{\textcolor{red}{b. Suites définies par une formule de récurrence}}}}
\begin{itemize}
\item[$\blacktriangleright$] Exemple :
\item[$\blacktriangleright$] Exercice d’application
\end{itemize}
\subsection*{\underline{\textbf{\textcolor{red}{4. Monotonie ou sens de variation d’une suite}}}}
\subsection*{\underline{\textbf{\textcolor{red}{a. Suite croissante}}}}
\begin{itemize}
\item[$\blacktriangleright$] Exemple :
\item[$\blacktriangleright$] Exercice d’application
\end{itemize}
\subsection*{\underline{\textbf{\textcolor{red}{b. Suite décroissante}}}}
\begin{itemize}
\item[$\blacktriangleright$] Exemple :
\item[$\blacktriangleright$] Exercice d’application
\end{itemize}
\subsection*{\underline{\textbf{\textcolor{red}{Remarque :}}}}
\section*{\underline{\textbf{\textcolor{red}{II.Suites arithmétiques:}}}}
\subsection*{\underline{\textbf{\textcolor{red}{1. Définition:}}}}
\begin{itemize}
\item[$\blacktriangleright$] Exemple :
\item[$\blacktriangleright$] Exercice d’application
\end{itemize}
\subsection*{\underline{\textbf{\textcolor{red}{2. Expression du terme général d’une suite arithmétique}}}}
\subsection*{\underline{\textbf{\textcolor{red}{a. Propriété 1}}}}
\begin{itemize}
\item[$\blacktriangleright$] Exemple :
\end{itemize}
\subsection*{\underline{\textbf{\textcolor{red}{a. Propriété 2}}}}
\begin{itemize}
\item[$\blacktriangleright$] Exemple :
\end{itemize}
\subsection*{\underline{\textbf{\textcolor{red}{3. Somme de termes consécutifs d’une suite arithmétique}}}}
\subsection*{\underline{\textbf{\textcolor{red}{a. Nombre de termes d’une somme de termes consécutifs}}}}
\subsection*{\underline{\textbf{\textcolor{red}{b. Somme de termes consécutifs d’une suite arithmétique}}}}
\begin{itemize}
\item[$\blacktriangleright$] Exemple :
\end{itemize}


\section*{\underline{\textbf{\textcolor{red}{III.Suites géométriques:}}}}
\subsection*{\underline{\textbf{\textcolor{red}{1. Définition:}}}}
\begin{itemize}
\item[$\blacktriangleright$] Exemple :
\item[$\blacktriangleright$] Exercice d’application
\end{itemize}
\subsection*{\underline{\textbf{\textcolor{red}{2. Expression du terme général d’une suite géométriques}}}}
\subsection*{\underline{\textbf{\textcolor{red}{a. Propriété 1}}}}
\begin{itemize}
\item[$\blacktriangleright$] Exemple :
\end{itemize}
\subsection*{\underline{\textbf{\textcolor{red}{a. Propriété 2}}}}
\begin{itemize}
\item[$\blacktriangleright$] Exemple :
\end{itemize}
\subsection*{\underline{\textbf{\textcolor{red}{3. Somme de termes consécutifs d’une suite géométriques}}}}
\begin{itemize}
\item[$\blacktriangleright$] Exemple :
\end{itemize}
\subsection*{\underline{\textbf{\textcolor{red}{4. Convergence d’une suite géométrique}}}}
\subsection*{\underline{\textbf{\textcolor{red}{c. Limites d’une suite}}}}
\subsection*{\underline{\textbf{\textcolor{red}{d. Propriété}}}}
\begin{itemize}
\item[$\blacktriangleright$] Exemple :
\end{itemize}
\end{document}