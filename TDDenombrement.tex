\documentclass[12pt]{article}
\usepackage{stmaryrd}
\usepackage{graphicx}
\usepackage[utf8]{inputenc}

\usepackage[french]{babel}
\usepackage[T1]{fontenc}
\usepackage{hyperref}
\usepackage{verbatim}

\usepackage{color, soul}

\usepackage{pgfplots}
\pgfplotsset{compat=1.15}
\usepackage{mathrsfs}

\usepackage{amsmath}
\usepackage{amsfonts}
\usepackage{amssymb}
\usepackage{tkz-tab}
\author{Destiné à la TL\\Au Lycée de Dindéferlo}
\title{\textbf{TD Dénombrement}}
\date{\today}
\usepackage{tikz}
\usetikzlibrary{arrows, shapes.geometric, fit}

% Commande pour la couleur d'accentuation
\newcommand{\myul}[2][black]{\setulcolor{#1}\ul{#2}\setulcolor{black}}
\newcommand\tab[1][1cm]{\hspace*{#1}}

\begin{document}
\maketitle
\newpage
\section*{\underline{\textbf{\textcolor{red}{Exercice 1}}}}
Deux ensembles $A$ et $B$ sont tels que : $\text{Card}(A)=24$ ; 
$\text{Card}(A\cup B)=46$ et $\text{Card}(A\cap B)=8$

Calculer le $\text{Card}(B)$ ?
\section*{\underline{\textbf{\textcolor{red}{Exercice 2}}}}
Une station de radio diffuse les mêmes publicités à $15$ heures et à $16$ heures.

D'après un sondage on sait qu'il y a $21400$ auditeurs ò $15$ heures et $24800$ à $16$ heures.

Combien de personnes ont entendu ces publicités :

1. si l'on suppose que les personnes qui ont écouté la radio à $15$ heures ne l'écoutent plus à $16$ heures ?
\section*{\underline{\textbf{\textcolor{red}{Exercice 3}}}}
Dénombrer les façons de composer un code de $4$ symboles sachant que les premier et troisième symboles sont des lettres et que les deuxième et quatrième symboles sont des chiffres.
\section*{\underline{\textbf{\textcolor{red}{Exercice 4}}}}
Lors d'une course olympique, $12$ coureurs s'affrontent pour trois médailles (Or, Argent et Bronze).
 
Combien y a-t-il de palmarès possibles ?
\section*{\underline{\textbf{\textcolor{red}{Exercice 5}}}}
On considère deux parties $A$ et $B$ d'un ensemble fini $E.$ Compléter les tableaux suivants :
$$\begin{array}{|l|c|c|c|c|c|c|c|c|c|} \hline \text{Ensemble}&E&amp;A&B&A\cap B&A\cup B&\overline{A}&\overline{B}&\overline{A\cap B}&\overline{A\cup B}\\ \hline \text{Card}&18&8&6&2&&&&&\\ \hline \text{Card}&100&75&&&&&50&75&\\ \hline \text{Card}&53&&34&&&23&&&8\\ \hline \end{array}$$
\section*{\underline{\textbf{\textcolor{red}{Exercice 6}}}}
Parmi $40$ secrétaires : 

$8$ parlent russe, 15 anglais et 8 espagnol
 
$4$ parlent anglais et espagnol
 
$5$ parlent anglais et russe
 
$2$ parlent espagnol et russe
 
$2$ parlent les trois langues.
 
a. combien de secrétaires parlent au moins une des trois langues ?
 
b. Combien de secrétaires ne connaissent aucune des trois langues ?

N.B : 

Faire un diagramme.
\section*{\underline{\textbf{\textcolor{red}{Exercice 7}}}}
Lors d'un examen, un candidat doit deux sujets : un en mathématiques (sur trois proposés) et un en français (sur quatre proposés).
 
Combien a-t-il de façons de choisir les deux sujets qu'il va traiter ?
\section*{\underline{\textbf{\textcolor{red}{Exercice 8}}}}
Soit $E$ un ensemble de $8$ éléments.

Combien existe-t-il :

a. d'arrangement de trois éléments de $E$ ?

b. d'arrangements de neuf élément de $E$ ?

c. de permutations de $E$ ?
\section*{\underline{\textbf{\textcolor{red}{Exercice 9}}}}
Combien existe-t-il d'anagrammes du mot THIES ? du mot AFRIQUE ?
\section*{\underline{\textbf{\textcolor{red}{Exercice 10}}}}
Combien existe-t-il de numéros de téléphones à six chiffres $6$ deux à deux
\section*{\underline{\textbf{\textcolor{red}{Exercice 11}}}}
Une classe de $20$ élèves veut élire un comité pour organiser  une excursion, comprenant un président, un vice-président, un trésorier et secrétaire.

Combien existe-t-il de comités possibles ?

	(on suppose que chaque élève est candidat à chaque poste, mais qu'il n'est pas permis de cumuler)
\section*{\underline{\textbf{\textcolor{red}{Exercice 12}}}}
Une urne contient $12$ boules.

Combien existe-t-il de façons de sortir $3$ boules, sachant qu'on extrait les boules une à une sans remise dans l'urne ?

Reprendre la question avec remise dans l'urne.
\section*{\underline{\textbf{\textcolor{red}{Exercice 13}}}}
ON dispose de $12$ cartons sur lesquels sont disposées les lettres $A$, $B$, $C$, $D$, $E$, $F$, $G$, $H$, $I$, $J$, $K$, $L.$ 

Combien y a-t-il de permutations où :

a. l'objet $A$ occupe la première place ?

b. Les objets $A$ et $B$ occupent les deux premières places ?
\section*{\underline{\textbf{\textcolor{red}{Exercice 14}}}}
On dispose de $6$ plaquettes sur lesquelles on inscrit les chiffres $2$, $4$, $5$, $6$, $8$ et $9.$

Combien de nombres distincts peut-on former en alignant ces $6$ plaquettes ?
\section*{\underline{\textbf{\textcolor{red}{Exercice 15}}}}
Astou a $4$ crayons de couleurs (bleu, jaune, rouge et vert) et veut dessiner un drapeau à $3$ bandes verticales.


Combien a-t-elle de possibilités pour dessiner un drapeau tricolore?
\section*{\underline{\textbf{\textcolor{red}{Exercice 16}}}}
Combien y a-t-il de façons d'organiser le passage de $6$ artistes pour un spectacle ?
\section*{\underline{\textbf{\textcolor{red}{Exercice 17}}}}
Soit les $5$ lettres $A$, $B$, $C$, $D$ et $E.$

a. Combien peut-on former de mots de $5$ lettres distinctes avec ces $5$ lettres ?

\textbf{Ivoire Svt} 

\textbf{Goualard} 

\textbf{Quelle} 

\textbf{Une} 

\textbf{Qui Est Matrix} 

b. Combien peut-on former de mot de $5$lettres distinctes tels que $A$ et $E$ ne soient pas voisines 

\section*{\underline{\textbf{\textcolor{red}{Exercice 18}}}}
On appelle « mot » toute permutation de lettres données.

Avec les lettres du mot BUNGALOW, combien peut-on former de mots :

a. de $8$ lettres ?

b. de $8$ lettres et commençant par deux consonnes ?

c. de $8$ lettres et commençant par deux voyelles ?

d. d $8$ lettres, commençant et finissant par une voyelle ?


e. de $8$ lettres, commençant par une consonne et finissant par une voyelle ?
\section*{\underline{\textbf{\textcolor{red}{Exercice 19}}}}
Cinq personnes $A$, $B$, $C$, $D$, $E$ et $F$ vont à un spectacle.

De combien de façons peut-on les faire asseoir sur six sièges cote à cote si :

a. $D$ insiste pour s'asseoir à coté de $A$ ?

b. $F$ refuse de s'asseoir  à cote de $B$ ?
\section*{\underline{\textbf{\textcolor{red}{Exercice 20}}}}
Combien de nombre de $4$ chiffres peut d'on former avec les chiffres $1$, $2$, $3$, $4$ et $5$ sachant que :

a. aucun chiffre ne doit être répété ?

b. les répétitions sont autorisées ?
\section*{\underline{\textbf{\textcolor{red}{Exercice 21}}}}
Combien de nombres inférieur à $2000$ peut-on fabriquer avec les chiffres $1$, $2$, $4$ et $7$ sachant que

a. aucun chiffre ne doit être répété ?

b. les répétitions sont autorisées.

Combien de ces nombres sont pairs ? Impairs ?
\section*{\underline{\textbf{\textcolor{red}{Exercice 22}}}}
Une urne contient $5$ boules blanches numérotées $B_{1}$,$\ldots$ $B_{5}$ et $4$ boules noires numérotées $N_{1}$,$\ldots$ $N_{4}.$

On tire successivement $4$ boules en remettant dans l'urne la boule tirée après chaque tirage.

Combien y'a-t-il de tirages distincts possibles ? 

Combien y’a-t-il e tirages unicolores ?

Combien y’a-t-il de tirages comportant autant de boules noires que de boules blanches ?

Combien y’a-t-il de tirages comportant plus de boules noires que de boules blanches ?

Reprendre les $4$ questions précédentes en supposant que le tirage se fait avec remise.
\section*{\underline{\textbf{\textcolor{red}{Exercice 23}}}}
On jette $3$ dés $A$, $B$, $C$ ayant chacun $6$ faces.
	
Combien y’a-t-il de résultats :
	
a. en tout ?
	
b. ou les $3$ faces sont identiques ?
	
c. ou les $3$ faces sont différents deux à deux
	
d. ne comportant aucun chiffre $1$ ?
	
e. comportant au moins un chiffre $1$ ?
	
f. comportant exactement un chiffre $1$ ?
	
g. comportant exactement deux chiffres $1$ ?
\end{document}