\documentclass[12pt]{article}
\usepackage{stmaryrd}
\usepackage{graphicx}
\usepackage[utf8]{inputenc}

\usepackage[french]{babel}
\usepackage[T1]{fontenc}
\usepackage{hyperref}
\usepackage{verbatim}

\usepackage{color, soul}

\usepackage{pgfplots}
\pgfplotsset{compat=1.15}
\usepackage{mathrsfs}

\usepackage{amsmath}
\usepackage{amsfonts}
\usepackage{amssymb}
\usepackage{tkz-tab}
\author{Destinés à la TLe\\Au Lycée de Dindéferlo}
\title{\textbf{TD Sur les Suites}}
\date{\today}
\usepackage{tikz}
\usetikzlibrary{arrows, shapes.geometric, fit}

% Commande pour la couleur d'accentuation
\newcommand{\myul}[2][black]{\setulcolor{#1}\ul{#2}\setulcolor{black}}
\newcommand\tab[1][1cm]{\hspace*{#1}}

\begin{document}
\maketitle
\newpage
\section*{\underline{\textbf{\textcolor{red}{Exercice 1}}}}
1. Soit $\left(U_{n}\right),\ n\in\mathbb{N}$, une suite arithmétique de raison $r=5$ et de premier terme $U_{1}$ telle que $U_{80}=393$.
	
Calculer $U_{1}$ et $S_{80}=U_{1}+U_{2}+\ldots+U_{80}$.
	
2. Déterminer le premier terme et la raison de la suite arithmétique $\left(U_{n}\right),\ n\geq 3$, telle que $U_{5}=17$ et $U_{7}=21$.
	
3. Soit $\left(U_{n}\right),\ n\in\mathbb{N}$, une suite arithmétique de raison $r$ et de premier terme $U_{1}=5$.
	
Déterminer l'entier $n$ et la raison telle que\\ $U_{n}=-16$ et $S_{n}=U_{1}+U_{2}+\ldots+U_{n}=-38.5$.
	
4. Soit $\left(U_{n}\right),\ n\in\mathbb{N}$, une suite géométrique de premier terme $U_{0}=4$ et de raison $q=\dfrac{1}{3}$.
	
Calculer $U_{6}$ et $S_{6}=U_{0}+U_{1}+\ldots+U_{6}$.
	
Soit $\left(U_{n}\right),\ n\in\mathbb{N}$, une suite géométrique de premier terme $U_{1}=2$.
	
Déterminer $q$ et $S_{4}=U_{1}+U_{2}+U_{3}+U_{4}$.
\section*{\underline{\textbf{\textcolor{red}{Exercice 2: BAC 2004 ; série L’1 :2e groupe}}}}
On considère la suite $(U_n)$ définie par :
\[
\begin{cases}
    U_0 = 9, \\
    U_{n+1} = 3U_n + 3, & n \geq 1.
\end{cases}
\]
1) Calculer $U_2$ et $U_3$.

2) On définit la suite $(V_n)$ par $V_n = U_n + \frac{3}{2}$, pour $n\geq1$
\begin{enumerate}
    \item[a)] Calculer $V_1$ et $V_2$.
    
    \item[b)] Montrer que $(V_n)$ est une suite géométrique et préciser le premier terme et la raison.
    
    \item[c)] Exprimer $V_n$ puis $U_n$ en fonction de $n$.
    
    \item[d)] Montrer que $(V_n)$ et $(U_n)$ sont divergentes.
\end{enumerate}
\section*{\underline{\textbf{\textcolor{red}{Exercice 3}}}}
La raréfaction d'une matière première oblige un pays à envisager d'en diminuer la consommation de $8\%$ par an. 
	
Celle-ci était en $1986$, $U_{0}=100$ (en millions de tonnes).
	
1. Calculer la consommation $U_{1}$ en $1987$ (c'est-à-dire au bout d'un an) et celle de $U_{2}$ en $1988$.
	
Exprimer en fonction de $n$ la consommation $U_{n}$ en l'an « $1986+n$ » (c'est-à-dire au bout de $n$ années).

2. En quelle année la consommation sera-t-elle pour la première fois inférieure à $1$ million de tonnes ?
	
3. Quel doit être le pourcentage de diminution annuelle imposé pour atteindre une consommation annuelle égale à $1$ million de tonnes en $20$ ans ?
\section*{\underline{\textbf{\textcolor{red}{Exercice 4}}}}
(Tous les résultats sont arrondis au nombre entier le plus proche)
	
En $1990$, la population d'une ville a enregistré $1000$ naissances et $900$ décès. 
	
Tous les ans, le nombre de naissances augmente de $8\%$ et le nombre de décès de $2\%$.
	
1. Quels ont été le nombre de naissances et de décès en $1991$ ; puis en $1992$ ?
	
2. Quels seront le nombre de naissances et de décès en $2050$ ?
	
3. Combien de naissances et de décès ont été enregistrés entre le $1^{er}$ Janvier $1990$ et le $31$ Décembre $1999$ ?
	
4. À partir de quelle année le nombre annuel de naissances aura-t-il doublé celui des décès ?
\section*{\underline{\textbf{\textcolor{red}{Exercice 5}}}}
La raréfaction d'une matière première oblige un pays à envisager d'en diminuer la consommation de $8\%$ par an.
	
Celle-ci était en $1986$, $U_{0}=100$ (en millions de tonnes).
	
1. Calculer la consommation $U_{1}$ en $1987$ (c'est-à-dire au bout d'un an) et celle de $U_{2}$ en $1988$.
	
Exprimer en fonction de $n$ la consommation $U_{n}$ en l'année $1986+n$ (c'est-à-dire au bout de $n$ années).
	
2. En quelle année la consommation sera-t-elle pour la première fois inférieure à 1 million de tonnes ?
	
3. Quel doit être le pourcentage de diminution annuelle imposé pour atteindre une consommation annuelle égale à 1 million de tonnes en 20 ans ?
\end{document}